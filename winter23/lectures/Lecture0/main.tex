\documentclass[10pt]{beamer}
\usepackage[utf8]{inputenc}

\usepackage{multirow,rotating}
\usepackage{color}
%\usepackage{hyperref}
\usepackage{tikz-cd}
\usepackage{array}
\usepackage{siunitx}
\usepackage{mathtools,nccmath}%
\usepackage{etoolbox, xparse} 

\usetheme{CambridgeUS}
\usecolortheme{dolphin}

% set colors
\definecolor{blue}{HTML}{005a8b}
\definecolor{brightblue}{HTML}{a0cfeb}
\definecolor{brighterblue}{HTML}{0157ad}
\definecolor{darkblue}{HTML}{003c66}
\definecolor{textcolor}{HTML}{000222}



\definecolor{warmgrey}{HTML}{988f86} % {130,138,143}
\setbeamercolor*{palette primary}{bg=darkblue}
\setbeamercolor*{palette secondary}{bg=blue, fg = white}
\setbeamercolor*{palette tertiary}{bg=warmgrey, fg = white}
\setbeamercolor*{titlelike}{fg=blue}
\setbeamercolor*{title}{bg=textcolor, fg = white}
\setbeamercolor*{item}{fg=textcolor}
\setbeamercolor*{caption name}{fg=warmgrey}
\usefonttheme{professionalfonts}
\usepackage{natbib}
%\usepackage{hyperref}
\usepackage[colorlinks=true,urlcolor=blue]{hyperref} 
%------------------------------------------------------------
% \titlegraphic{\includegraphics[height=0.75cm]{logo.png}} 

% logo of my university




\setbeamerfont{title}{size=\Huge}
\setbeamerfont{subtitle}{size=\Large}
\setbeamerfont{author}{size=\small}
\setbeamerfont{date}{size=\large}
\setbeamerfont{institute}{size=\footnotesize}
\title[PSTAT 126: Regression Analysis]{PSTAT 126}%title
\subtitle{Regression Analysis }%%subtitle
\author[Laura Baracaldo \& Rodrigo Targino]{Laura Baracaldo \& Rodrigo Targino}%%authors

\institute[UCSB]{}
\date[\textcolor{white}{Lecure 0: Course Introduction}]
{Lecture 0\\
Course Introduction}

%------------------------------------------------------------
%This block of commands puts the table of contents at the 
%beginning of each section and highlights the current section:
%\AtBeginSection[]
%{
%  \begin{frame}
%    \frametitle{Contents}
%    \tableofcontents[currentsection]
%  \end{frame}
%}
\AtBeginSection[]{
  \begin{frame}
  \vfill
  \centering
  \begin{beamercolorbox}[sep=8pt,center,shadow=true,rounded=true]{title}
    \usebeamerfont{title}\insertsectionhead\par%
  \end{beamercolorbox}
  \vfill
  \end{frame}
}
% ------Contents below------
%------------------------------------------------------------

\begin{document}

%The next statement creates the title page.
\frame{\titlepage}



% consider removing it if it's too redundant


%------------------------------------------------------------

\begin{frame}{Course Content}

\begin{itemize}
\large
    \item This course introduces the theory and application of linear regression models. 
    \item Topics: simple and multiple regression models; estimation; inference; prediction; regression diagnostics; model selection; shrinkage methods; analysis of variance. \item R: Solving real-world problems.
\end{itemize}

\end{frame}



\begin{frame}{Class Format}
\large
{\bf Canvas/Gauchospace}
\begin{itemize}

\item Lecture slides and labs will be available on {\em Canvas/Gauchospace} and also here:
\url{shorturl.at/fgrt8}
\item You will need to close and open again the link to fetch newly added content.
\item Homework assignments and quizzes will also be given out on {\em Canvas/Gauchospace}, and should be turned in there. Other forms of submission will not be accepted.
\item All Q\&A related to course content, homework assignments, R programming, and quizzes should
be done on the nectir channel \url{https://app.nectir.io/invite/frdqQD}.
\end{itemize}
\end{frame}

\begin{frame}{Office hours}
	\large
	\begin{itemize}
		\item Prof. Baracaldo: TBD
		\item Prof. Targino: Fridays 1.30pm - 3.30pm OG1230 and on Zoom \url{https://fgv-br.zoom.us/j/95030256507}
	\end{itemize}
\end{frame}

\begin{frame}{References}
\large
 The lecture slides are self-contained. You may find the following textbooks helpful:

\begin{itemize}
\item Faraway, J. J. (2005),\textit{ Linear Models with R}, Chapman \& Hall.
\item Weisberg, S. (2005),\textit{ Applied Linear Regression}, 3rd edition, Wiley.
\end{itemize}  

\vspace{0.05in}

{\bf R programming}
\begin{itemize}
\item \textit{R for Data Science} by Grolemund and Wickham.    \url{https://r4ds.had.co.nz/index.html}
\end{itemize}
\end{frame}

\begin{frame}{Grading}
\large
\begin{itemize}
   \item {\bf 40 \%}. Homeworks. (Four hw assignments, due every two weeks.)
   \item {\bf 20 \%}. Quizzes (On canvas)
   \item {\bf 40 \%}. Final exam
\end{itemize}
    
\end{frame}

\begin{frame}{Homework}

\begin{itemize}
\large
   \item R coding \& Math deriving/proof.
   \item Typically you will have 2 weeks to complete the homework.
   \item Late homework will receive 20\% point deduction.
   \item Homework will not be accepted more than 24 hrs late.
\end{itemize}
    
\end{frame}

\begin{frame}{R, Rstudio \& Rmarkdown}

\large
\href{https://cloud.r-project.org}{\bf R}. User\-friendly programming language for statistical analysis.\\
\href{http://www.rstudio.com/download}{\bf R studio}. Integrated development environment (IDE) for R.\\
 {\bf R markdown (Rmd)}. Generates reproducible document with R.

 \begin{itemize}
     \item Syntax is very simple.
     \item Integrated with LaTeX, easy to write math formulas.
     \item Example: HomeworkTemplate.Rmd. (on \texttt{/homeworks/Template} on the server)
 \end{itemize}
 {\em Task: } Start setting up the R working environment (either in your own computer or in the server)!
\end{frame}

\begin{frame}{Homework submission Format}
\large
\begin{itemize}
    \item Online submission via Canvas/Gauchospace
    \item Submission must contain: R markdown code (.Rmd file), PDF generated, supplementary files if needed.
    \item We should be able to run R markdown code to obtain identical PDF file.
\end{itemize}


    
\end{frame}

\begin{frame}{Next...}
\large
\begin{itemize}
    \item Make sure you can knit HomeworkTemplate.Rmd.
    \item Lab sessions start on the 2nd week.
    \item Read from {\em R for Data Science}: Ch 1\&2 (very short), Ch 27 on (Rmd).
\end{itemize}
    
\end{frame}

\end{document}